\documentclass[11pt,a4paper]{article}

\usepackage{tabularx}
\usepackage[utf8,latin1]{inputenc}
\usepackage[T1]{fontenc}
\usepackage[spanish]{babel}

\usepackage{geometry}
\usepackage{array}

\title{Trabajo Pr\'actico Ingenier\'ia 2}
\author{
	Dami\'an Alem\'an \and
	Ignacio Harari
	Mart\'in Arjovsky \and
	Mart\'in Fixman \and
}
\date{Primer Cuatrimestre de 2016}

\begin{document}
\maketitle
\newpage

\newgeometry{margin=1in}

\section{Casos de Uso}

\begin{flushleft}
\begin{tabularx}{\textwidth}{@{} | X | >{\raggedleft\arraybackslash}p{4em} | >{\raggedleft\arraybackslash}X | @{}}
\hline
\textbf{Caso de Uso} & \textbf{Tiempo} & \textbf{Justificaci\'on} \\ \hline
Armando Equipo & 16 hs & \\ \hline
Armando Desaf�o & 40 hs & \\ \hline
Viendo progreso de desaf�o & 8 hs &  \\ \hline
Aceptando desaf�o & 6 hs &  \\ \hline
Mandando mensaje privado & 30 hs &  \\ \hline
Mandando mensaje general & 15 hs & Significativamente implementado en el punto 5 \\ \hline
Ingresando m�todo de pago & 20 hs & Riesgo por seguridad y plata \\ \hline
Viendo cuenta regresiva & 4 hs &  \\ \hline
Viendo partido en vivo & 40 hs &  \\ \hline
Accediendo a datos de usuario y de marketing & 20 hs & Esto es riesgoso por el punto de vista de la anonimizaci�n de los usuarios \\ \hline
Extendiendo datos de simulaci�n & 20 hs & \\ \hline
Accediendo al log minuto a minuto & 20 hs &  \\ \hline
Visualizando simulaci�n de partido & 40 hs &  \\ \hline
Incorporando estad�sticas de redes sociales y mensajes & 60 hs & Interpretar mensajes es altamente no trivial, y hay un riesgo por anonimidad \\ \hline
Visualizando ranking & 25 hs &  \\ \hline
Accediendo estado de la cuenta & 20 hs &  \\ \hline
Restringiendo acceso/features seg�n legislaci�n & 20 hs & Puede haber problemas legales \\ \hline
Retirando fondos & 30 hs & Hay que considerar los posibles casos donde puede haber fraude financiero \\ \hline
\end{tabularx}
\end{flushleft}

\vspace{1em}

Esto nos da un total de \textbf{434 horas hombre} de trabajo.

\restoregeometry

\section{Cronograma}

\begin{itemize}
	\item {\LARGE\bf Primera iteraci\'on (121 horas hombre)}
	\begin{itemize}
		\item Armando equipo
		\item Armando desaf�o
		\item Ingresando m�todo de pago
		\item Visualizando simulaci�n de partido
		\item Accediendo al log minuto a minuto
	\end{itemize}

	\item {\LARGE\bf Segunda iteraci\'on (106 horas hombre)}
	\begin{itemize}
		\item Mandando mensaje privado
		\item Aceptando desaf�o
		\item Accediendo estado de la cuenta
		\item Retirando fondos
		\item Extendiendo datos de simulaci�n
	\end{itemize}

	\item {\LARGE\bf Tercera iteraci\'on (100 horas hombre)}
	\begin{itemize}
		\item Mandando mensaje general
		\item Viendo partido en vivo
		\item Visualizando ranking
		\item Restringiendo acceso/features seg�n legislaci�n
	\end{itemize}

	\item {\LARGE\bf Cuarta iteraci\'on (92 horas hombre)}
	\begin{itemize}
		\item Accediendo a datos de usuario y de marketing
		\item Incorporando estad�sticas de redes sociales y mensajes
		\item Viendo progreso de desaf�o
		\item Viendo cuenta regresiva
	\end{itemize}
\end{itemize}

La primera iteraci�n fue elegida teniendo en cuenta que los tiempos son
razonables y que la funcionalidad de cada caso de uso es importante por las
siguientes razones.

\begin{itemize}
  \item Armando equipo y armando desaf�o son la parte central de la
    aplicaci�n. Adem�s de eso, el frontend de estos casos de uso es
    importante para ver con los stakeholders y puede requerir futuros
    cambios. Armando desaf�os adem�s incluye la parte de poner una apuesta
    en el desaf�o, cosa que puede interesar a los stakeholders dado que es
    una potencial fuente de ganancia.
  \item Ingresando m�todo de pago fue puesta en la primera iteraci�n por dos
    razones. Primero, para mostrar a los stakeholders que la funcionalidad
    m�nima necesaria para incorporar dinero a la aplicaci�n (y una potencial fuente
    de ingresos) est� siendo incorporada y testeada lo antes posible. Segundo,
    debido a su alto riesgo, al requerir mucha seguridad para evitar fraude 
    financiero, es importante que sea estudiada y testeada desde un principio,
    para evitar potenciales pitfalls.
  \item Vizualizando simulaci�n de partido es una funcionalidad radicalmente 
    nueva en la aplicaci�n, y visualmente impactante, por lo que es �til de tener
    en una primera demo a los stakeholders.
  \item Accediendo al log minuto a minuto pertenece al core de la funcionalidad
    y los logs pueden ser de ayuda cr�tica al debugear siguientes etapas del
    desarrollo, adem�s de ser un caso de uso de relativamente poco tiempo de
    desarrollo.
\end{itemize}

\section{Casos de uso con riesgo}

Hay varios casos de uso que consideramos que son riesgosos, dado que tienen complicaciones que deben ser revisadas con especial cuidado.

En particular,

\begin{description}
	\item [Accediendo los datos de los usuarios] es riesgoso por un tema de privacidad. Seg�n la especificaci�n dada por los stakeholders se piden los datos personales de los usuarios sin anonimizar, y esto puede traer ramificaciones legales.
	\item [Ingresando metodo de pago] tiene que tener una implementaci�n que sea totalmente segura, o hay riesgo de que se cometa fraude y que haya problemas legales serios. Hay que tener especial cuidado a ataques que est�n dirigidos al robo de informaci�n.
	\item [Incorporando estad�sticas de redes sociales y mensajes] puede ser riesgoso ya que un usuario puede personificar a uno de estos servicios para mejorar si standing en el juego.
	\item [Restringiendo acceso/features seg�n legislaci�n] es riesgoso porque un error o la mala interpretaci�n de una ley puede llevar a problemas legales inmediatos.
	\item [Viendo partido en vivo] es riesgoso desde el punto de vista de la disponibilidad, el streaming es un proceso demandante de recursos que es relativamente propenso a fallas.
	\item [Retirando fondos] es riesgoso porque hay una alta posibilidad de que usuarios intenten hacer fraude mediante este servicio.
\end{description}

\end{document}
