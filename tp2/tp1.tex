La segunda versi\'on de la aplicaci\'on claramente tiene una gran cantidad de 
componentes que no se encuentran en la primera, ya que simplemente contiene m\'as
 features.

Entre los componentes que ambas comparten, una de las primeras
 decisiones de arquitectura que fueron necesariamente diferentes fue la del
Simulador, debido a la necesidad de crear un log minuto a minuto para el 
funcionamiento de los nuevos graficadores, tuvimos que modificar la estructura
de tal manera que el simulador pudiera dar paso a paso los resultados de las 
jugadas al logger y al graficador.

Por fuera de lo que es el simulador, el 
Calculador de Popularidad tambi\'en sufri\'o algunos cambios debido las redes 
sociales que se agregaron al an\'alisis, junto con los datos descifrados de los 
mensajes de usuarios, sin embargo eso no se vio tan reflejado en la 
arquitectura del componente, impactando en cambio en mayor medida a la 
implementaci\'on de el Scraper y el Interprete que su organizaci\'on 
en la arquitectura.



Otros componentes que son cualitativamente distintos entre ambas versiones son 
el Controlador de Desaf\'ios, y los componentes pertenecientes al flujo de las 
apuestas.

En cuanto a los desaf\'ios, en la segunda versi\'on adem\'as debe estar a 
cargo de manejar el flujo de los torneos que son ahora posibles, con el 
componente encargado de manejar las cuentas regresivas colaborando en ello. Por 
otro lado, los componentes relacionados al flujo de las apuestas fueron 
ampliamente simplificados en la primer versi\'on de la aplicaci\'on (reducidos a una
 substracci\'on y adici\'on de cr\'editos) ya que \'esta no estaba monetizada, en cambio
 en la segunda versi\'on fue necesario tener en cuenta los flujos para a\'nadir 
cr\'editos, lo cual requiere una transacci\'on (pago) de dinero, y el de convertir 
los cr\'editos nuevamente a dinero, ambos m\'as complejos por tratarse de 
transacciones con dinero real.
