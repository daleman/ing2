La segunda versión de la aplicación claramente tiene una gran cantidad de componentes que no se encuentran en la primera, ya que simplemente contiene más features.

Entre los componentes que ambas comparten, una de las primeras decisiones de arquitectura que fueron necesariamente diferentes fue la del Simulador, debido a la necesidad de crear un log minuto a minuto para el funcionamiento de los nuevos graficadores, tuvimos que modificar la estructura de tal manera que el simulador pudiera dar paso a paso los resultados de las jugadas al logger y al graficador.

Por fuera de lo que es el simulador, el Calculador de Popularidad también sufrió algunos cambios debido las redes sociales que se agregaron al análisis, junto con los datos descifrados de los mensajes de usuarios, sin embargo eso no se vio tan reflejado en la arquitectura del componente, impactando en cambio en mayor medida a la implementación de el Scraper y el Interprete que su organización en la arquitectura.

Otros componentes que son cualitativamente distintos entre ambas versiones son el Controlador de Desafíos, y los componentes pertenecientes al flujo de las apuestas.

En cuanto a los desafíos, en la segunda versión además debe estar a cargo de manejar el flujo de los torneos que son ahora posibles, con el componente encargado de manejar las cuentas regresivas colaborando en ello. Por otro lado, los componentes relacionados al flujo de las apuestas fueron ampliamente simplificados en la primer versión de la aplicación (reducidos a una substracción y adición de créditos) ya que ésta no estaba monetizada, en cambio en la segunda versión fue necesario tener en cuenta los flujos para añadir créditos, lo cual requiere una transacción (pago) de dinero, y el de convertir los créditos nuevamente a dinero, ambos más complejos por tratarse de transacciones con dinero real.
