\subsection{UP vs. Agile}

UP utiliza el concepto de time boxing y una separaci\'on de las tareas de 
acuerdo a sus caracter\'isticas que permite m\'as claramente atacar los riesgos 
esperados y las calidades deseadas del software de manera temprana en el 
proceso de desarrollo, gui\'andose siempre por los casos de uso definidos 
durante el desarrollo (y en especial al principio del proyecto). 
Uno de los resultados m\'as prioritarios del proceso es la arquitectura, lo 
cual afecta la manera y el orden en el que se afrontan las tareas.

En las metodolog\'ias \'agiles, hay un mayor foco en que el resultado de cada 
iteraci\'on sea un producto ``andando'',
se centra en tener iteraciones 
relativamente cortas que permitan una gran interacci\'on entre el cliente y el 
equipo de desarrollo, para poder guiar (y redireccionar cuando sea necesario) 
la direcci\'on del producto. A diferencia de UP, el factor que gu\'ia la 
prorizaci\'on de las tareas suele ser su ``bussiness value''.


\subsection{Dise\~{n}o OO vs. Arquitectura}

En lo que se refiere a dise\~{n}o OO, muchas veces el centro de atenci\'on se 
encuentra en la caracter\'istica del software como objeto cambiante, y por ende 
muchas de las pr\'acticas usuales se dirigen a facilitar ese cambio, priorizando 
que el c\'odigo producido sea funcional (debe ser una buena representaci\'on del 
dominio modelado, para facilitar plasmar cambios y caracter\'isticas del dominio) 
y descriptivo (un modelo dif\'icil de entender es inherentemente dif\'icil de 
modificar), en cambio el dise\~{n}o de arquitectura, si bien tambi\'en puede ayudar a 
guiar un desarrollo que cumpla con esas normas, tiene en su eje proveer un plan 
que permita minimizar los riesgos y maximizar las calidades deseadas, 
describiendo los elementos del sistema y las caracter\'isticas de sus relaciones 
en tanto permita especificar propiedades requeridas para asegurar los aspectos 
de calidad que los ``stakeholders''
definan a lo largo del proceso de desarrollo. 
Es v\'alido aclarar que no son procesos disjuntos y es posible (y deseable) 
utilizar ambos para encaminar el proceso de desarrollo de una manera que ataque 
tanto los riesgos y calidades deseadas, y que facilite una implementaci\'on 
descriptiva y maleable.
