\begin{itemize}
	\item La \textbf{p\'agina web} le pide al controlador de partidos la simulaci�n v�a un client server. Controlador de partido realiza la simulaci�n pidiendo la data necesaria al calculador de puntos. Luego escribe el resultado en el log minuto a minuto y le manda los pedidos de video al controlador de engine. Este �ltimo decide si hacer video 2d o 3d y pide el video al engine correspondiente. Acto seguido, controlador de engine pone el video simulado en un pipe al controlador de partidos para que se los mande a la p�gina web.
	\item El \textbf{Sistema de Pagos y Cobros} se ocupa de la comunicaci\'on del sistema con el banco y con la tarjeta de cr\'edito para cobrarle dinero al usuario (y obtener cr\'editos) o devolverle dinero que puede haber comprado o ganado en apuestas (y, asumiendo que tenga los cr\'editos suficientes, cobrarselos).
	\item El \textbf{Calculador de Puntos} calcula los puntos correspondientes a una jugada con informaci\'on obtenida del Control de Partido. Este usa f\'ormulas sacadas de la base de datos, y le pide datos de los jugadores en las redes sociales al \textit{Calculador de Puntos de Popularidad}.
	\item El \textbf{Controlador de Popularidad} tiene la responsabilidad de buscar en las redes sociales informaci\'on sobre qu\'e est\'a diciendo la gente sobre cada jugador, as\'\i se puede saber cu\'antos puntos se deben agregar. Esta informaci\'on es recolectada por el \textit{Scraper}, que consigue datos de \textit{Facebook} y \textit{Twitter}, y enviada al \textit{Int\'erprete de Texto}, que devuelve una serie de ``keywords'' al Controlador.
\end{itemize}
