\begin{itemize}
	\item La \textbf{P\'agina Web} le pide al controlador de partidos la simulaci�n v�a un client server. Controlador de partido realiza la simulaci�n pidiendo la data necesaria al calculador de puntos. Luego escribe el resultado en el log minuto a minuto y le manda los pedidos de video al controlador de engine. Este �ltimo decide si hacer video 2d o 3d y pide el video al engine correspondiente. Acto seguido, controlador de engine pone el video simulado en un pipe al controlador de partidos para que se los mande a la p�gina web.
	\item El \textbf{Sistema de Pagos y Cobros} se ocupa de la comunicaci\'on del sistema con el banco y con la tarjeta de cr\'edito para cobrarle dinero al usuario (y obtener cr\'editos) o devolverle dinero que puede haber comprado o ganado en apuestas (y, asumiendo que tenga los cr\'editos suficientes, cobrarselos).
	\item El \textbf{Calculador de Puntos} calcula los puntos correspondientes a una jugada con informaci\'on obtenida del Control de Partido. Este usa f\'ormulas sacadas de la base de datos, y le pide datos de los jugadores en las redes sociales al \textit{Calculador de Puntos de Popularidad}.
	\item El \textbf{Controlador de Popularidad} tiene la responsabilidad de buscar en las redes sociales informaci\'on sobre qu\'e est\'a diciendo la gente sobre cada jugador, as\'\i se puede saber cu\'antos puntos se deben agregar. Esta informaci\'on es recolectada por el \textit{Scraper}, que consigue datos de \textit{Facebook} y \textit{Twitter}, y enviada al \textit{Int\'erprete de Texto}, que devuelve una serie de ``keywords'' al Controlador.
	\item El \textbf{Controlador de Engine} es activado por el \textit{Controlador de Partido}, que tambi\'en le env\'{\i}a las propiedad del sistema para saber si debe usar el \textit{Engine 3D} \'o el \textit{Engine 2D}. Dependiendo de la informaci\'on prove\'{\i}da por este y lee el \textit{Log Minuto a Minuto} para crear la simulaci\'on gr\'afica del partido, que env\'{\i}a de vuelta al control del partido.
	\item El \textbf{Apostador} recibe datos de apuestas de un usuario desde la p\'agina web y de sus cr\'editos. Luego de terminado un partido, este cobra o agrega cr\'editos al usuario.
	\item El \textbf{Administrador de Desaf\'{\i}os} administra los partidos que todav\'{\i}a no fueron concretados, y que los usuarios pueden aceptar o no. El \textit{Administrador Web} le comunica que hay un nuevo desaf\'{\i}o o que uno ya existente fue aceptado, y en el \'ultimo caso lo guarda en la base de datos de desaf\'{\i}os y activa la cuenta regresiva. Cuando este termina, le env\'{\i}a al \textit{Controlador de Desaf\'{\i}os} que el partido debe comenzar, y este le adelanta la notificaci\'on al \textit{Controlador de Partido}.
\end{itemize}
