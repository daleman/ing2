\documentclass{beamer}

\mode<presentation>
{%
	\usetheme{Warsaw}
	\setbeamercovered{invisible}
}
\setbeamertemplate{caption}{\raggedright\insertcaption\par}

\usepackage[spanish]{babel}
\usepackage[utf8]{inputenc}

\title{Managing the Development of Large Software Systems}

\author[Pipe \& Filter]{%
	{\Large Pipe \& Filter} \\ \vspace{1em}
	Martín Fixman\inst{1} \and
	Ignacio Harari\inst{1} \and \\
	Damián Alemán\inst{1} \and
	Martín Arjovsky\inst{1}
}
\institute{\inst{1} Facultad de Ciencias Exactas y Naturales}

\date{Primer Cuatrimestre 2016}

\begin{document}

\begin{frame}
\titlepage{}
\end{frame}

\section{Introducción}

\begin{frame}{Introducción}

Esta presentación demuestra algunas observaciones sobre la administración de proyectos como presentado en el paper del {\large Dr.\ Winston W.\ Royce}\cite{royce70}.

\bigskip

En este, se presenta un proceso para mejorar los pasos a seguir durante la administración de un proyecto grande para prevenir errores y lograr bajar los costos de corregirlos cuando ocurren.

\end{frame}

\end{document}
