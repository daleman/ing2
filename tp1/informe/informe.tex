\documentclass[11pt,a4paper]{article}
\usepackage[utf8]{inputenc}
\usepackage{a4wide}
\usepackage{fancyhdr}
\pagestyle{fancy}
\thispagestyle{fancy}

\usepackage{gensymb}
\usepackage{hyperref}
\newcommand\fnurl[2]{%
  \href{#2}{#1}\footnote{\url{#2}}%
}
\usepackage{amsfonts}
\usepackage{natbib}

\usepackage[spanish]{babel}
\parskip = 11pt

\usepackage{tabularx}

\usepackage[conEntregas]{caratula}
\materia{Ingenier\'ia del Software 2}
\titulo{Trabajo Pr\'actico I}
\subtitulo{\textit{Curry Game}}
\integrante{Aleman, Dami\'an Eliel}{377/10}{damianealeman@gmail.com}
\integrante{Arjovsky, Mart\'in}{}{martinarjovsky@gmail.com}
\integrante{Fixman, Mart\'in}{}{martinfixman@gmail.com}
\integrante{Harari, Ignacio}{415/11}{nachotee@hotmail.com}

\def\tabularxcolumn#1{m{#1}}

\begin{document}

\maketitle

\section*{User Stories}

Las User Stories mostradas en \textbf{negrita} son las ser\'an completadas en el primer Sprint.

\begin{table}[h]
\begin{tabularx}{\textwidth}{|X|c|c|}
\hline
\textbf{Descripci\'on} & \textbf{Esfuerzo} & \textbf{  Value  } \\ \hline
Como participante quiero poder registrar una cuenta para acceder al juego. & \Large2 & \Large8  \\ \hline
Como participante quiero buscar desafíos para poder aceptarlos. & \Large5 & \Large5 \\ \hline
Como participante quiero armar equipos para poder aceptar o postular desafíos con ese equipo. & \Large3 & \Large5 \\ \hline
Como administrador quiero administrar jugadores disponibles para poder organizar el juego. & \Large3 & \Large2 \\ \hline
Como administrador quiero consultar las cuentas que están en el sistema para poder analizar el uso del juego por los participantes. & \Large5 & \Large2\\ \hline
Como participante quiero autenticar mi cuenta (log in) para poder entrar en el sistema de juego. & \Large2 &\Large8 \\ \hline
Como participante quiero poder ver mi cap y mis fichas para decidir mis apuestas. &\Large2 &\Large3 \\ \hline
Como participante quiero ver tabla de posiciones de todos los participantes en base a sus desafíos ganados/perdidos para poder conocer el ranking. & \Large3 & \Large2\\ \hline
Como participante quiero postular desafío para desafiar a otros participantes a que jueguen contra mi equipo. & \Large2 & \Large5 \\ \hline
Como participante quiero aceptar el desafío postulado por otro participante para poder jugar contra otros equipos. & \Large2 & \Large5 \\ \hline
Como administrador quiero poder crear jugadas ofensivas y defensivas. &\Large5&\Large3 \\ \hline

\textbf{Como administrador quiero poder simular una acción ofensiva y defensiva.} &\Large5 & \Large8 \\ \hline
\textbf{Como administrador quiero editar fórmulas y números  de resolución de acciones para poder cambiar el juego en caso de que se necesite.} & \Large5 & \Large3\\ \hline
\textbf{Como participante quiero ver log del partido e información pertinente para poder visualizar las simulaciones anteriores.} & \Large3 &\Large8 \\ \hline
\textbf{Como participante quiero nombrar mi equipo, elegir mi jugador estrella y mi tecnico.} & \Large2&\Large5 \\ \hline
\end{tabularx}
\end{table}

\end{document}
