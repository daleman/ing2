\section{Seguimiento}

\subsection{Contraataque}

Representar al contraataque dentro de la aplicaci�n, por tener caracter�sticas especiales que no comparte con el resto de las estrategias ofensivas, fue desde un principio un problema. Antes de llegar a la soluci�n final que elegimos, ya explicada anteriormente, discutimos otras que terminamos descartando por diversos motivos. Por ejemplo, una de las alternativas inclu�a mantener al contraataque como una \textbf{EstrategiaOfensiva}, y crear entonces dos \textbf{Acciones Ofensivas} nuevas que representaran un \textbf{Tiro Inbloqueable} (de 2 y 3 puntos). Los problemas de tener esas acciones, y de modelar al contraataque como una estratgeia ofensiva (en el contexto de nuestro modelo) eran varios, en principio, el contraataque como estrategia deb�a exluirse de poder ser elegido por un t�cnico como estrategia ofensiva, y de modelarlo con Tiros Inbloqueables, habr�a requerido una seccion especial en el m�todo de \textbf{simular} que chequeara luego de cada intercepci�n si deb�amos inicializar un contra ataque o no, y la manera de elegir si deb�amos hacer un contraataque o no, no nos era del todo clara ya que no estaba especificada. En pos de mantener al m�todo de \textbf{simular} lo m�s gen�rico posible y agn�stico el tipo de Jugada que se est� simulando, decidimos cambiar un poco la interpretaci�n de la manera que ya fue explicada, y encapsular el comportamiento especial en su propio m�todo del simulador.