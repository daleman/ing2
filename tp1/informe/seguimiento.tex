\section{Seguimiento}

Aquí explicaremos algunas de las problemáticas que nos fuimos encontrando al resolver el trabajo, las alternativas que nos planteamos, las soluciones en las que concluímos y las razones de éstas decisiones.

\subsection{Contraataque}

Representar al contraataque dentro de la aplicación, por tener características especiales que no comparte con el resto de las estrategias ofensivas, fue desde un principio un problema. Antes de llegar a la solución final que elegimos, ya explicada anteriormente, discutimos otras que terminamos descartando por diversos motivos. Por ejemplo, una de las alternativas incluía mantener al contraataque como una \textbf{EstrategiaOfensiva}, y crear entonces dos \textbf{Acciones Ofensivas} nuevas que representaran un \textbf{Tiro Inbloqueable} (de 2 y 3 puntos). 

Los problemas de tener esas acciones, y de modelar al contraataque como una estrategia ofensiva (en el contexto de nuestro modelo) eran varios, en principio, el contraataque como estrategia debía exluirse de poder ser elegido por un tícnico como estrategia ofensiva, y de modelarlo con Tiros Inbloqueables, habría requerido una seccion especial en el método de \textbf{simular} que chequeara luego de cada intercepción si debíamos inicializar un contra ataque o no, y tendríamos que haber decidido la manera de elegir si debíamos hacerlo o no, que no nos era del todo clara ya que no estaba especificada. En pos de mantener al método de \textbf{simular} lo más genérico posible y agnóstico el tipo de Jugada que se esté simulando, decidimos cambiar un poco la interpretación de la manera que ya fue explicada, y encapsular el comportamiento especial en su propio método del simulador.

Una solución alternativa que no implementamos por su complejidad, pero que podría haber sido más genérica (y entonces, más modificable) de haberla pensado hasta el final era la de tener una serie de \textbf{ModificadoresDeExito} que pudieran agregarse a las \textbf{Acciones}, y que serían llamados para modificar la probabilidad de éxito desde la \textbf{Acción} (en el momento en el que ésta calcula su triunfo), junto con una suerte de \textbf{Tabla de Efectividad}, que sería un objeto encargado de saber la efectividad de una estretegia ofensiva particular contra otra estrategia defensiva, y viceversa, devolviendo los \textbf{Modificadores} correspondientes. Esto nos habría permitido modelar el contraataque agregando a la Tabla la regla de que, cuando se realiza un contraataque, las acciones defensivas deben llevar un modificador que las hace inefectivas (ya que así está definido en el enunciado), y nos habría permitido definir otros modificadores de la aplicación, como es el cansancio que resulta de ejecutar una jugada MvP de manera similar.

\subsection{Simulación}

La simulación del partido es claramente la parte más complicada de la aplicación, desde el punto de vista de diseño, por lo que tuvimos que iterar sobre muchas soluciones fallidas hasta llegar a algo satisfactorio, la clave aquí fue el Double Dispatch. El inconveniente se encontraba en, dada una estrategia ofensiva, y una defensiva, necesitabamos saber exáctamente con qué \textbf{Accion Defensiva} responder a la \textbf{Acción Ofensiva} que estuvieramos analizando en ese momento; dentro de las opciones posibles barajamos la idea de tener una suerte de Map que funcionara como oráculo al cual, al darle las estrategias, nos devolviera los objetos correctos para continuar la simulación, pero si bien eso encapsularía la decisión, requeriría tener \textbf{if's} dentro de ella de alguna manera, con lo cual al llegar a la solución de Double Dispatch concluimos entre todos que era un mucho mejor diseño.


\section{Retrospectiva General}
En líneas generales después de realizar el product backlog se nos hizo muy difícil
realizar las tareas a tiempo como estaba estipulado para lograr un burndownchart
con una velocidad relativamente constante, que sería lo ideal. 
Sin embargo creemos que esto no es debido a la metodología de trabajo ya que 
ésta realiza retrospectivas de forma constante justamente para que 
no ocurran estas problemáticas de tiempos.
Al ser un trabajo práctico, es decir un trabajo que no se realiza a diario 
es más difícil tener la constancia necesaria
para realizar adecuadamente estos tipos de metodologías ágiles.
